In this section we describe the structure of the configuration file to be used
to complete a seismic hazard disaggregation. Since only a few parts of the
standard configuration file need to be changed we can use the description
given in Section~\ref{subsec:config_classical_psha} at
page~\pageref{subsec:config_classical_psha} as a reference and we emphasize
herein major differences.


\begin{minted}[firstline=1,linenos=true,firstnumber=1,fontsize=\footnotesize,frame=single,bgcolor=lightgray]{ini}
[general]
description = A demo .ini file for PSHA disaggregation
calculation_mode = disaggregation
random_seed = 1024
\end{minted}

The calculation mode parameter in this case is set as
\texttt{disaggregation}.



\textbf{Geometry of the area (or the sites) where hazard is computed}

\begin{minted}[firstline=1,linenos=true,firstnumber=5,fontsize=\footnotesize,frame=single,bgcolor=lightgray]{ini}
[geometry]
sites = 11.0 44.5
\end{minted}

In the section it is necessary to specify the geographic coordinates of
the site (or sites) where the disaggregation will be performed.



\textbf{Disaggregation parameters}

\begin{minted}[firstline=1,linenos=true,firstnumber=7,fontsize=\footnotesize,frame=single,bgcolor=lightgray]{ini}
[disaggregation]
poes_disagg = 0.02, 0.1
mag_bin_width = 1.0
distance_bin_width = 25.0
coordinate_bin_width = 1.5
num_epsilon_bins = 3
\end{minted}

With the disaggregation settings shown above we'll disaggregate the intensity
measure levels with 10\% and 2\% probability of exceedance using the
\texttt{in\-ves\-ti\-gation\_time} and the intensity measure types  defined in
the ``Calculation configuration'' section of the OpenQuake configuration file
(see page~\pageref{sec:calculation_configuration}).

The parameters \texttt{mag\_bin\_width},  \texttt{distance\_bin\_width},
\texttt{coordinate\_bin\_width} control the level of discretization of the
disaggregation matrix computed. \texttt{num\_epsilon\_bins} indicates the
number of bins used to represent the contributions provided by different
values of epsilon.

It is also possible to perform disaggregation by directly specifying a single
intensity measure level to be disaggregated. This can be done by replacing
\texttt{poes\_disagg} and associated probabilities with \texttt{iml\_disagg}
and the associated intensity measure level (g). This is a useful feature if
a user wants to perform disaggregation of a hazard model that contains multiple
logic tree branches and the user is interested in disaggregating, for example,
the mean hazard or a specific quantile. An example of a possible workflow is
the following: compute the desired hazard curve (mean or quantile) using the
classical PSHA hazard calculator, extract from the curve the intensity measure
corresponding to the desired hazard level to be disaggregated, then specify
that intensity measure level using \texttt{iml\_disagg}. Note that
\texttt{poes\_disagg} cannot be set if \texttt{iml\_disagg} is set and vice-versa. 
An example is shown below:

\begin{minted}[firstline=1,linenos=true,firstnumber=5,fontsize=\footnotesize,frame=single,bgcolor=lightgray]{ini}
[disaggregation]
iml_disagg = 0.3
\end{minted}

If the user is interested in a specific type of disaggregation, we suggest to
use a very coarse gridding for the parameters that are  not necessary. For
example, if the user is interested in a magnitude-distance  disaggregation, we
suggest the use of very large value for the
\texttt{coordinate\_\-bin\_\-width} and to set  \texttt{num\_epsilon\_bins}
equal to 1.
